\documentclass[11pt]{article}
\usepackage[table]{xcolor}
\definecolor{gray-empty}{gray}{1.0}
\definecolor{gray-conceptualize}{gray}{0.75}
\definecolor{gray-design}{gray}{0.5}
\definecolor{gray-implement}{gray}{0.25}
\definecolor{gray-support}{gray}{0}
\usepackage{tcolorbox}
\newcommand\timelinebox[1]{%
  \tcbox[size=tight,on line,colframe=gray-#1,colback=gray-#1]{\textcolor{gray-#1}{XXX}}
}
\usepackage{setspace}
\usepackage{caption}
\usepackage{subcaption}
\usepackage{fancyvrb}
\usepackage{epsfig}
\usepackage{fullpage}
\usepackage[small,compact]{titlesec}
\usepackage{tcolorbox}
\usepackage{upquote}
\usepackage{xcolor}
\usepackage{soul}
\tcbuselibrary{minted, skins, listings}
\tcbset{listing engine=minted}

\newtcbinputlisting{pyoutput}[2]{
  listing file=sections/examples/#2,
  listing and comment,
  minted style = emacs,
  minted language = python,
  left=0cm,
  right=0cm,
  top=0cm,
  bottom=0cm,
  comment={#1}}

\newtcbinputlisting{pyimage}[2]{
  listing file=sections/examples/#2,
  listing side comment,
  minted style = emacs,
  minted language = python,
  righthand width=5.5cm,
  left=0cm,
  right=0cm,
  top=0cm,
  bottom=0cm,
  lower separated=false,
  tcbimage comment={sections/examples/#1},
  comment style={size=fbox}}

% This may not always be necessary:
% http://tex.stackexchange.com/questions/299969/titlesec-loss-of-section-numbering-with-the-new-update-2016-03-15
\usepackage{etoolbox}

\makeatletter
\patchcmd{\ttlh@hang}{\parindent\z@}{\parindent\z@\leavevmode}{}{}
\patchcmd{\ttlh@hang}{\noindent}{}{}{}
\makeatother
% End

\usepackage{hyperref}
%\usepackage{times}

\renewcommand*{\sectionautorefname}{\S}
\renewcommand*{\subsectionautorefname}{\S}
\usepackage{wrapfig}
\usepackage{enumitem}
\setlist{nolistsep}
\setlength{\topmargin}{+0.0in}   %%%%%%%%% this is hacked (from +.0.1in) so that it looks right when converted.
\setlength{\oddsidemargin}{-0.0in}
\setlength{\evensidemargin}{-0.0in}
\setlength{\textheight}{9.0in}
\setlength{\textwidth}{6.5in}
\bibliographystyle{plain}
\newcommand{\apj}{ApJ}
\newcommand{\apss}{Astrophysics and Space Science}
\newcommand{\aj}{AJ}
\newcommand{\apjl}{ApJL}
\newcommand{\mnras}{MNRAS}
\newcommand{\apjs}{ApJS}
\newcommand{\pasp}{PASP}
\newcommand{\pasa}{PASA}
\newcommand{\araa}{ARA\&A}
\newcommand{\aap}{A\&A}
\newcommand{\aaps}{A\&AS}
\newcommand{\pasj}{PASJ}
\newcommand{\prd}{Phys. Rev. D}
\newcommand{\nat}{Nature}
\newcommand{\physrep}{Physics Reports}
\newcommand{\etal}{et al}
\newcommand{\yt}{\texttt{yt}}
\newcommand{\libyt}{\texttt{libyt}}
\newcommand{\visit}{\texttt{VisIt}}
\newcommand{\paraview}{\texttt{ParaView}}
\newcommand{\mayavi}{\texttt{Mayavi}}
\newcommand{\grackle}{\texttt{Grackle}}
\newcommand{\enzo}{\texttt{Enzo}}
\newcommand{\dengo}{\texttt{Dengo}}
\newcommand{\vtk}{\texttt{VTK}}
\newcommand{\sympy}{\texttt{SymPy}}
\def\subsun{\mbox{$_{\odot}$}}
\def\lesssim{\mathrel{\hbox{\rlap{\hbox{%
 \lower4pt\hbox{$\sim$}}}\hbox{$<$}}}}
\def\gtrsim{\mathrel{\hbox{\rlap{\hbox{%
 \lower4pt\hbox{$\sim$}}}\hbox{$>$}}}}
\newcommand\YTEP[1]{%
  YTEP-#1}
\renewcommand\YTEP[1]{YTEP-#1}

\usepackage{natbib}
\setlength{\bibsep}{0.0pt}

\newenvironment{itemize*}%
{\begin{itemize}%
  \setlength{\itemsep}{0pt}%
    \setlength{\parskip}{0pt}}%
{\end{itemize}}

\newenvironment{enumerate*}%
{\begin{enumerate}%
  \setlength{\itemsep}{0pt}%
    \setlength{\parskip}{0pt}}%
{\end{enumerate}}

\newenvironment{description*}%
{\begin{description}%
  \setlength{\itemsep}{0pt}%
    \setlength{\parskip}{0pt}}%
{\end{description}}

\begin{document}

\clearpage

% http://www.nsf.gov/pubs/2016/nsf16532/nsf16532.htm
%
% Management and Coordination Plan (SSI proposals only, 3-page limit, to be
% submitted as a Supplementary Document): Each SSI proposal must contain a
% clearly labeled Management and Coordination Plan that includes: 1) the
% specific roles of the PI, co-PIs, other Senior Personnel and paid consultants
% at all institutions involved; 2) how the project will be managed across
% institutions and disciplines; 3) identification of the specific coordination
% mechanisms that will enable cross-institution and/or cross-discipline
% scientific integration (e.g., yearly workshops; graduate student exchanges;
% project meetings at conferences; use of video conferences; use of common
% software repositories, build processes and/or test suites; etc.); and 4)
% pointers to the budget line items that support these management and
% coordination mechanisms.

\clearpage
\pagenumbering{arabic}
\setcounter{page}{1}

\begin{center}
%\bfseries{%
%%
%% ENTER TITLE OF PROPOSAL BELOW THIS LINE
%{\large Inquiry-Focused Volumetric Data Analysis Across Scientific
%  Domains:\\Sustaining and Expanding the \yt{} Community}\\
  {\large \textbf{MANAGEMENT AND COORDINATION PLAN}}
%%
%%
%}
\end{center}

\section{Roles of Individuals Involved}

Below, we identify the specific roles that each Investigator and Senior
Personnel member will occupy.

\subsection{University of California - San Diego}

The University of Illinois will be the primary coordinating institution, and
will function as the primary developers of the \grackle{} interface and its
community support.

\noindent \textbf{Britton Smith} (PI): Smith will coordinate broad goals
and conduct development, as well as act as a liaison and advisor for liaising
between communities of researchers.  He will also participate in development of
the \grackle{} code, interfacing that with generated code from \dengo{}, and
work to unify the disparate code bases.  Development of accelerator code for
\grackle{} will proceed under his guidance, and the students at UCSD will be
working with Smith.  Smith will also organize the advisory board meetings.

\subsection{University of Illinois}

Primary development of the \dengo{} infrastructure will be conducted at UIUC.

\noindent \textbf{Matthew Turk} (PI):  Turk will coordinate with Smith and
other community members to ensure functionality in \grackle{} can be
sufficiently replicated in \dengo{}-generated code.  Additionally, he will
supervise students working on symbolic reaction rate generation and integration
into simulation codes.

\section{Project Management and Coordination Mechanisms}

UCSD will serve as the coordinating institution.
The advisory board will conduct
virtual meetings no less than once a year, with much more frequent (monthly or
more) progress reports and discussions occuring over asynchronous methods of
communication such as email.

\grackle{} itself has a number of existing project mechanisms which will
be augmented as part of this proposal.  This includes issue management, mailing
lists and discussions.

We note here that by default, \textit{all} communication regarding this project
will be public.  This includes goal-setting, milestone evaluation, and
evaluation of components.

\section{Budget Support}

We identify the travel section, which will fund annual visits between research
groups, as budget items in support of this management plan.  We will be
utilizing cloud-hosted open source solutions, most of which are well within the
``free" tier, for communication.

\end{document}

%%% Local Variables:
%%% mode: latex
%%% TeX-master: t
%%% End:
