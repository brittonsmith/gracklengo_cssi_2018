\section{Results from Prior Support}

\noindent \textbf{Britton Smith} (UCSD PI). NSF AST-1615848 (PI: Michael
Norman): \textit{Collaborative Research: CDS\&E: Renaissance Simulations
Laboratory to Model and Explore the First Galaxies in the Universe},
\textbf{Amount:} \$440,064
\textbf{Period:} 08/01/2016-07/31/2019,
\textbf{Products:} Jupyter notebooks and prototypes are available at
\url{https://girder.rensimlab.xyz/}.  Publications:
\citep{2016ApJ...832L...5X, 2016ApJ...833...84X, Barrow17_FL2,
2017ApJ...845...47T, 2017ApJ...847...59H}.
\textbf{Intellectual Merit:} This grant enables further analysis of
the Renaissance Simulations \citep{2015ApJ...807L..12O} and creates
an open-access data analysis portal, known as the ``Renaissance Simulation
Laboratory'' (RSL), to provide the astronomical community with access
to the raw data, analysis data products, and tools of the Renaissance
Simulations.
%% The RSL provides a data analysis workflow that allows users to
%% run custom analysis on any of the hosted data and to download and/or
%% share their data products and scripts.
\textbf{Broader impacts:}
We provide open access to this set of cutting-edge
simulations and associated tools to the entire astronomical community,
independent of a user's local computing resources.
%% This both
%% increases the return on investment in the simulations and helps to
%% level the playing field for scientific research with large data sets.

\noindent \textbf{Matthew Turk} (UIUC PI).  
NSF ACI-1535651: \textit{yt: Reusable Components for Simulating,
Analyzing and Visualizing Astrophysical Systems},
\textbf{Amount}: \$511,989.00
\textbf{Period}: 11/01/2014-9/30/2018,
\textbf{Products}: \yt{} source code
\textbf{Intellectual Merit}: The method paper for \yt{} has been cited over
260 times and the source code has been used to dramatically advance the speed
and detail of research in astrophysical sciences.
\textbf{Broader Impacts}: The developments from this grant have materially
contributed to analysis results and visualizations used in numerous
astrophysical papers. \yt{} has additionally been used in the development
of planetarium shows for education and public outreach.
