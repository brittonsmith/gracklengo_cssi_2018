\section{Community Growth}\label{sec:community_growth}

\bds{Need a more specific plan for community engagement!}

Long before it was known by that name, the \grackle{} source code has
been developed as community software as part of the \texttt{Enzo}
code, with significant contributions made by the PI.  The \grackle{}
library was created to fulfill a need
for a cross-platform tool that would enable comparison of the many
different simulation codes in the AGORA project.  PI Smith led the
effort, starting in late 2012, to generalize \texttt{Enzo}'s machinery
and transform it into this tool.  PI Smith also worked with members of
the AGORA project to ascertain what additional functionality would be
required and then made these additions.  Since then, the PI has worked
to grow the \grackle{} community beyond the context of AGORA, and in
that time the number of codes with \grackle{} support grew from the
original 9 in the AGORA project to at least 14.

Strengthening community engagement is critical to the success of the
\grackle{} project after the lifetime of this SSE.  PI Smith has
significant experience organizing scientific software projects as a
core developer of the \texttt{Enzo} code and core developer and
steering committee member of the \yt{} project.  PI Smith authored
\yt{}'s original project governance documentation (YT Enhancement
Proposal 1776) that established the steering committee and codified the
project's procedures for development, team organization, and
acknowledgement of significant contributions.  PI Smith has worked to
model the \grackle{} community after the successes of \yt{} and
\texttt{Enzo}.

The growth of the \grackle{} community will also involve an
educational component in the form of instruction and communication of
new features (such as the new network added in year three) to users,
and training and mentorship for developers, especially after the
significant infrastructure upgrades occuring in years one and two.  PI
Smith has extensive experience in this role, as an instructor at
multiple \texttt{Enzo} and \yt{} users workshops both in the US and
abroad, and in many one-on-one tutorials of \grackle{}, \texttt{Enzo},
and \yt{} with students, at the undergraduate and graduate levels, and
postdocs.  In the summer of 2014 at the Royal Observatory of
Edinburgh, PI Smith organized and taught a series of lectures on
Python skills for scientific computing that was open to all in the
department.  This SSE will provide funding to host a
workshop after year two to provide instruction for new developers in
general best practices and code optimization for hardware
accelerators.  The SSE will also provide funding for team members to
attend conferences and workshops to present on \grackle{} and make
connections to new users.  In these efforts, we will make use of the
lessons and techniques promoted by the Software
Carpentry\footnote{http://www.software-carpentry.org/} project.

This SSE is an investment in the \grackle{} project that will enable
the development projects necessary for making the code a
self-sustaining community resource.  Each component of the work has
been designed to address a critical factor for long term success of
the code: the ease with which it can be developed, its performance
using the latest technologies, and the broadness of its
applicability.
