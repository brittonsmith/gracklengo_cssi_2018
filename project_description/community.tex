\section{Community Growth}\label{sec:community_growth}

Before it was known by that name, the \grackle{} source code had
been developed as community software as part of the \texttt{Enzo}
code, with significant contributions made by the PIs and collaborator
Greg Bryan.  The \grackle{} library was created to be cross-platform
tool that would enable comparison of the
different simulation codes in the AGORA project; not only is it one of the only
chemical kinetic libraries that has support for multiple codes, it is among the
\textit{only} pieces of software in the astrophysical simulation community that
has seen wide uptake by multiple independent groups.  In late 2012, PI
Smith started the effort to generalize \enzo{'s} machinery into a library.
PI Smith worked with members of
the AGORA community to ascertain what additional functionality would be
required and responded to those needs with solutions that worked with
all the supported codes.  Since then, PI Smith has worked to grow the
\grackle{} community beyond the context of AGORA through direct
engagement with members of the simulation community.  The number of
simulation codes using \grackle{} has grown to at least 14 and the
code has received 76 PRs from 17 different contributors, including PI
Turk and collaborators Bryan and Glover.

Strengthening community engagement is critical to the success of the
\grackle{} project after the lifetime of this award.  Both PIs
have more than a decade of experience cultivating user and developer
communities around open-source software projects, including
\grackle{}, the \enzo{} simulation code, and the \yt{} analysis
toolkit (an SI2-funded project.)  PI Turk is the \yt{} project founder
and both PIs serve on its steering committee.  They are also both core
developers of the \enzo{} code.  Their activities have included
crafting extensive documentation and tutorial materials; fostering
discussion and providing help over mailing lists; reaching out to new
contributors and in-person mentoring; and organizing and participating
in numerous user and developer workshops for both \enzo{} and \yt{}.
Experience has shown that in-person workshops are particularly
effective for community building as they allow newcomers to interact
directly with more experienced user/developers who can provide
real-time help and guidance.  Similarly, developer workshops both
strengthen the sense of community and provide the opportunity to
advance collaborative development projects that ultimately result in
the release of new code features.

To help grow the \grackle{} community, we will host user/developer
workshops at the end of years two and three at the Center for
Computational Astrophysics (CCA) at the Flatiron Institute in New York
City, where collaborator Bryan has a partial appointment.  The CCA has
committed to provide a venue and services for two 20-30 person
workshops (see attached collaboration letter).  The CCA provides an
ideal venue as it hosts numerous researchers in computational
astrophysics whose needs can be served by \grackle{}.  Like prior
\enzo{} and \yt{} workshops, these will consist of two user days of
tutorials and hands-on instruction and two developer days of code
sprints and planning discussions.  The instructional materials
resulting from these workshops, including tutorial videos and example
Jupyter notebooks, will be integrated into the \grackle{}
documentation. For crafting tutorials, we will make use of the lessons
and techniques promoted by Software
Carpentry\footnote{http://www.software-carpentry.org/}.  These materials will
provide not only technical information about using \grackle{}, but background
information and exploratory experience with astrochemistry and how
astrophysical plasmas evolve under different conditions.
