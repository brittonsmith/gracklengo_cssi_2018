\section{Broader Impacts}

The proposed work will have a number of important broader impacts that go
beyond its technical achievements.  First, it will provide equal
access to functionality that is necessary for simulation or modelling
research for a wide range of astrophysical topics, including the
formation and evolution of stars, galaxies, black holes, and the
intergalactic medium.
Second, the \texttt{pygrackle} Python module's ease of use lowers the
barrier to entry to using methods that have traditionally only been
available in the context of large, complex simulations.  By
encouraging community contribution, \grackle{} gives researchers
a direct channel to all users of the codes that have adopted the
\grackle{} API, thus providing an avenue for dissemination of
work and for forming collaborations.  This will enable exploration of topics
for researchers at all stages of their career, deepening the intellectual connections between
theory and application of chemistry.  These benefits will have the
greatest impact on those with less well-developed professional
networks, who tend to be earlier-stage researchers or those from
underrepresented groups.  This will enable the \grackle{} project
to continue to promote a diverse and inclusive community.

By providing this critical functionality equally to simulation codes
that use different methods, \grackle{} plays a key role in the comparison of
results, a fundamental pillar of the scientific method.  Evidence of
this is already available in the form of the AGORA simulation
comparison project \citep{2014ApJS..210...14K, 2016ApJ...833..202K},
which continues to use \grackle{} as a core component.

As a well-tested, thoroughly documented library with a well-designed
structure and clear API, \grackle{} also provides an example of the
use best practices for software development and computing.  This is
furthered by the guidance and mentorship that less experienced
developers receive through reviews of Pull Requests and
advice on the mailing list.  These skills can be applied far
beyond computational astrophysics, in other sciences and industry.  Integration
with the technologies underpinning \dengo{}, such as jinja2, sympy and ODE
solver systems, will help to integrate researchers into a modern software stack
that is being used across disciplines, from software engineering and web
development to data science and theoretical mathematics.
