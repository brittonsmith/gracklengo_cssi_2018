\section{The Many Challenges of Chemistry Solving}

The formation and evolution of astrophysical structures, such as
galaxies and molecular clouds, are governed
by a combination of non-linear phenomena, most notably gravitation,
hydrodynamics, and the atomic-scale physics that determine the
efficiency with which a plasma expels energy via radiative and
chemical processes.  Within simulation codes, gravity and
hydrodynamics solvers are relatively low-maintenance components, in
that they very rarely require updates due to advances in our
understanding of their processes.  Scaling is the only barrier to
increased sophistication.
In contrast, solvers for chemical and radiative processes are
extremely high-maintenance components, as the field of study they
represent evolves rapidly and is multifaceted, with advances coming
from theoretical calculations \citep[e.g.,][]{2007MNRAS.377..705F,
  2007MNRAS.382..133W, 2008MNRAS.388.1627G, 2008ApJ...689.1105L,
  2012JChPh.137o4303L, 2014ApJ...790...10S, 2015MNRAS.453..810L,
  2016MNRAS.457.3732C, 2017MNRAS.466.2175C}, laboratory experiments
\citep[e.g.,][]{2010Sci...329...69K, 2010PhRvA..82d2708B,
  2011PhRvA..84e2709M, 2015JPhCS.635b2092R, 2015ApJS..219....6O,
  2016ApJ...816...31D, 2016ApJ...832...31V}, and even improved integration
methods \citp[e.g.,]{CURTIS2017312, 2014JCoPh.256..854N}.  As a result, there
has been considerable divergence in the methods employed for solving chemistry
and cooling within the astronomical community.  This hinders scientific
progress by making it difficult to directly compare results from
different simulations and by mandating that considerable effort
within each research group be devoted keeping methods up to date.  The
\grackle{} chemistry and cooling library was developed with the
explicit purpose of addressing these problems, and its growing popularity
demonstrates its clear value to the community.

Gas chemistry presents a unique set of challenges in terms of
both computation and software development.  First, the network of
reactions rapidly grows in complexity with the number of elements and
species of those elements (i.e., the various ionization states and
molecular forms) considered.  For example, the simplest case of a
primordial gas consisting of hydrogen, deuterium, and helium minimally
requires between 11 and 15 species and roughly 20 to 30 reactions
\citep{1997NewA....2..181A, 1998A&A...335..403G}.
Adding just carbon and oxygen increases the minimal network to 
roughly 15-30 species and more than 50 reactions
and potentially up to almost 500, depending on the desired level of
accuracy \citep{2005ApJ...626..627O, 2012MNRAS.421..116G}.  Increases in
computational power will continue to be easily consumed by adopting
even gradually more sophisticated chemical networks and optimization
will remain critical.  An even greater challenge is that the
knowledge base
informing these solvers is continually evolving.  The rate
coefficients for many important reactions are still highly uncertain
\citep{2008MNRAS.388.1627G, 2011ApJ...726...55T} with accepted values
changing with new results from experimentation
\citep{2010Sci...329...69K, 2015ApJS..219....6O, 2016ApJ...816...31D}.
Additionally, new models are frequently created to account for
phenomena that are difficult to resolve or simulate directly, such as
UV background models that mimic reionization
\citep[e.g.,][]{1996ApJ...461...20H, 2001cghr.confE..64H,
2012ApJ...746..125H, 2009ApJ...703.1416F}, atomic
\citep{2013MNRAS.430.2427R} and molecular \citep{1996ApJ...468..269D,
2012MNRAS.425L..51W} self-shielding from photo-ionizing/dissociating
radiation, and heavy element cooling tables for metal-enrichment from
multiple sources \citep[e.g.,][]{2009MNRAS.393...99W,
2013MNRAS.433.3005D}.

Thus, a chemistry and cooling solver
represents, at best, a snapshot of an ever-evolving body of
knowledge.  The functionality provided by chemistry and cooling solvers is
absolutely vital to simulations and models of a broad range of
astrophysical problems: galaxy formation \citep{2016ApJ...830L..13M,
2016MNRAS.462.3265D, 2017MNRAS.465.2540P, 2017MNRAS.466..105A,
2017MNRAS.tmp..110D}; disk galaxies \citep{2015MNRAS.449.2588P,
2015ApJ...814..131G}; dwarf galaxies \citep{2017arXiv170108779H}, 
star formation \citep{2013RvMP...85.1021T, 2016Natur.535..523F,
2017ApJ...835..137W};
planet formation \citep{2013RvMP...85.1021T, 2017A&A...605A..16F,
2018arXiv180310526A};
black hole formation \citep{2016MNRAS.459.3217L, 2016MNRAS.459.4209A,
2016MNRAS.459.3377R, 2016MNRAS.461..111R}; supernovae remnants
\citep{2012ApJ...748...12S, 2016arXiv161008528B, 2017MNRAS.465.2471G};
the interstellar medium \citep{2015ApJ...814....4L,
2016arXiv161201786K}; the intergalactic medium
\citep{2011ApJ...731....6S, 2011MNRAS.413..190T, 2012MNRAS.420..829O};
and reionization \citep{2014ApJ...789L..32K, 2015ApJ...811....3S}.
The tradition throughout the astrophysical simulation community has
been to develop in-house solvers, geared to the specific design of
each code and baked directly into the source, making abstraction and
direct comparison between codes extremely difficult.
