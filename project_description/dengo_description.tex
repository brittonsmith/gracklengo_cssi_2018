\section{\dengo{}: Solver Generation}

A computational solver for chemical kinetic rate equations is the synthesis of
three components: 1) a set of reactions and their affiliated species,
2) a collection of reaction rate coefficients and 3) the mechanism of updating
the state vector from one time to a future time.  Ensuring that a collection of
reaction rate coefficients is up to date with current theoretical or
experimental values is straightforward; this requires tracking changes in the
literature and modifying the input parameters to interpolation tables.
Adding a single reaction requires modifying the right-hand side
calculation for all species that appear in that reaction.  This poses
challenges to ensuring bookkeeping between reaction rates and species are
correct, and poses entirely new challenges when the solver itself is built into
the reaction network.  The two primary types of solvers used in astrophysical
calculations are explicit (and semi-implicit) solvers and fully-implicit
solvers, each with their (dis)advantages.  Writing an explicit solver is an
entirely different process than writing a fully-implicit solver and so
transferring code -- and ensuring consistent levels of optimization -- between
solver methodologies is non-trivial.  An example of this is the Gauss-Seidel
update of the reaction state vector \textit{during} the timestep update, as is
done in \grackle{}.

Advances in open source computer algebra systems (CAS) such as SymPy
\citep{10.7717/peerj-cs.103} enable the construction
of reaction rate networks that can be symbolically and numerically
differentiated as well as output into a templated format.  For instance,
constructing a reaction network composed of a few dozen chemical species and a
few hundred reactions can be accomplished in only as many lines of code as is
necessary to describe them.  These reaction networks can have Jacobians (exact
or approximate) computed and they preserve the notion of an abstract model that
can be instantiated into one or more different concrete representations of a
chemical solver.

%% When a reaction network requires scaling to a large number of species
%% or a large number of reactions (particularly when the connectivity of the
%% network is high) the complexity of modifying this reaction network increases
%% rapidly.

%% While strictly separable from the other two components, often the solver itself
%% is intertwined between them.
%% The construction of a solver, and the inherent
%% bookkeeping, is prone to error, time-consuming, and is inflexible for simple
%% transformations between solver methodologies; writing an explicit or
%% semi-implicit solver is an entirely different process than writing a
%% fully-implicit solver, and the mixture of the model with the implementation is
%% non-trivial.  Furthermore, when interconnecting multiple species in non-linear,
%% often non-trivial ways, calculation of a Jacobian requires complex and
%% extensive differentiation.

%% Generating a reaction network from this involves constructing a change vector,
%% $\frac{\partial n_i}{\partial t}$ for all pairwise combinations and sets of
%% reactions.  From the perspective of implementation, this requires considerable
%% bookkeeping for each additional species and each additional reaction.  For
%% example, adding a single reaction rate that operates on two species requires
%% updating the species change vectors for each of those two species; when
%% optimizing for computational architectures -- such as SIMD instructions, memory
%% management for interpolation tables, and reducing cache misses -- this adds
%% not-inconsiderable overhead for the developer.  Furthermore, operations that
%% change the mechanism by which the solver operates, such as changing from
%% semi-implicit updates to a fully-implicit scheme, require full reorganization
%% of the solver code.  

%% At present, \grackle{} is a hand-written code, where the order of update, solver
%% prescriptions, interpolation of rates and the timestep estimate are all
%% intertwined with each other.  This reduces its overall flexibility, but it does
%% provide an opportunity for fine-grained and situation-aware optimization such
%% as update ordering.  However, it does not provide the flexibility, automated
%% bookkeeping and solver interoperability as would be provided through alternate,
%% symbolically defined reaction systems.

To address the limitations in \grackle{'s} flexibility, we will
further develop the \dengo{} package, a solver construction kit
designed by PI Turk. \dengo{} is built on a combination of SymPy and
the templating library Jinja2. \dengo{} includes full reaction
networks for primordial gas, ionization of atoms, and formation of molecules using UMIST
\citep{2013A&A...550A..36M}.  Constructing a solver using \dengo{} is done by
choosing a collection of species to evolve, choosing the relevant chemical
kinetic and radiative cooling equations that interlink them, and then
identifying the solver that will update the coupled differential equations at
each timestep. At present, \dengo{} can generate C code with Python
interfaces, utilizing either a backwards-Euler iterative solver or the SUNDIALS
suite of tools; furthermore, it has been recently instrumented to generate
OpenMP-aware code.

Figure \ref{fig:dengo} displays a sample application of \dengo{}, where
collaborator Devin Silvia used \dengo{} to generate a network for solving
the ionization states of atomic oxygen.  The network was coupled to
the \enzo{} simulation code to run a cosmological simulation of a 25
Mpc/h box with 256$^{3}$ grid cells and dark matter particles.  The
simulation was designed to study the observable signatures of the gas
between galaxies, known as the intergalactic medium (IGM).  These
results show that the amount of observable oxygen in the IGM is
significantly increased when NEQ effects are considered.
While this is promising, the inclusion of the ionization network
required roughly five times the computational resources, thus
highlighting the need for further optimization with hardware
accelerators as discussed below in the development plan.  It is worth
noting that the sole other attempt at a simulation with NEQ
O \citep{2016MNRAS.460.2157O} was done by running the
simulation to more than 80\% of completion under equilibrium first,
highlighting the genuine difficulty of this problem.

\dengo{} development has, thus far, been targeted at uses within a
single simulation code, \enzo{}, and so its API has not yet
stabilized.  By modifying \dengo{} to conform to \grackle{'s} API, the
\dengo{} project can take advantage of \grackle{'s} thriving user
community. Utilizing the existing \grackle{}
solver as a reference implementation, we will unify the two projects.
\dengo{} accepts reaction networks, chemical species (and their
metadata), reaction rates, and an API or implementation of an
iterative solver, and generates templated output code.  \grackle{}
continues to host the simplifying models (tabulated cooling, UV
backgrounds, etc.), but also supplies a curated set of
\dengo{}-generated solvers and provides a pathway for users to expand
them.

%% By unifying these two projects, we will bring to \textit{existing} users of
%% \grackle{} the ability to construct flexible networks that solve either a subset
%% of the typical reactions or a larger superset, and that are able to utilize
%% higher-order solvers for situations when reactions become particularly stiff.
%% Unifying the APIs will ensure that \textit{no additional code changes} are
%% necessary for existing simulation developers.
