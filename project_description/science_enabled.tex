\section{Science Enabled}

A number of areas in astrophysical structure formation exist where
theoretical progress has been limited by the computational expense of
solving complex chemical reaction networks or the unavailability of
key features in publicly available codes, like \grackle{}.  In most
cases, these phenomena can be studied using many of the simulation
codes in which the \grackle{} API is already supported.

\noindent \textbf{The Initial Mass Function of Primordial Stars}
The first stars in the Universe formed out of gas clouds composed only
of elements produced in the Big Bang, namely, H and He along with trace
amounts of D and Li.  These stars have yet to be observed, and so
predictions of their properties (i.e., masses, multiplicity, etc.) are
derived mainly from hydrodynamic simulations that follow the collapse
of primordial gas to high densities.  Despite the relatively simple
chemical composition, these simulations are still limited by the
computational expense of calculating the thermal evolution in the
dense cores, where chemical timescales become extremely short and
radiation transport effects become important.  Modern simulations fall
short of reaching stellar core densities by many orders of magnitude,
at which time the chemistry solve is the dominant cost in a given
timestep.

\noindent \textbf{Star Formation in the Local Universe}
Local star formation (need to discuss why complex networks are
essential for this, linking to observations or star forming regions,
etc.)

\noindent \textbf{Dust Formation in Supernova Remnants}

\noindent \textbf{Chemical Evolution}
Tracking individual abundances in galaxy formation simulations to
connect with Milky Way/Local Group stellar abundances/stellar archaeology.

\noindent \textbf{The Thermal Structure and Metal Content of the
  Intergalactic Medium}
Non-equilibrium ionization in IGM/CGM. Point to the few existing
papers and how they are very obviously limited by how expensive this
is.

\noindent \textbf{Planet Formation}
