\documentclass[11pt]{article}
\usepackage{color}
\usepackage{setspace}
\usepackage{caption}
\usepackage{subcaption}
\usepackage{fancyvrb}
\usepackage{epsfig}
\usepackage{fullpage}
\usepackage[small,compact]{titlesec}
\usepackage{hyperref}
%\usepackage{times} 
\usepackage{wrapfig}
\usepackage{enumitem}
\setlist{nolistsep}
\setlength{\topmargin}{+0.0in}   %%%%%%%%% this is hacked (from +.0.1in) so that it looks right when converted.
\setlength{\oddsidemargin}{-0.0in}
\setlength{\evensidemargin}{-0.0in}
\setlength{\textheight}{9.0in}
\setlength{\textwidth}{6.5in}
\bibliographystyle{apj}
\newcommand{\apj}{ApJ}
\newcommand{\apss}{Astrophysics and Space Science}
\newcommand{\aj}{AJ}
\newcommand{\apjl}{ApJL}
\newcommand{\mnras}{MNRAS}
\newcommand{\apjs}{ApJS}
\newcommand{\pasp}{PASP}
\newcommand{\pasa}{PASA}
\newcommand{\araa}{ARA\&A}
\newcommand{\aap}{A\&A}
\newcommand{\aaps}{A\&AS}
\newcommand{\pasj}{PASJ}
\newcommand{\prd}{Phys. Rev. D}
\newcommand{\nat}{Nature}
\newcommand{\physrep}{Physics Reports}
\newcommand{\etal}{et al}
\newcommand{\grackle}{\texttt{Grackle}}
\newcommand{\yt}{\texttt{yt}}
\def\subsun{\mbox{$_{\odot}$}}
\def\lesssim{\mathrel{\hbox{\rlap{\hbox{%
 \lower4pt\hbox{$\sim$}}}\hbox{$<$}}}}
\def\gtrsim{\mathrel{\hbox{\rlap{\hbox{%
 \lower4pt\hbox{$\sim$}}}\hbox{$>$}}}}
\RequirePackage{natbib}
%\usepackage{setspace}
\setlength{\bibsep}{0.0pt}

\newenvironment{itemize*}%
{\begin{itemize}%
  \setlength{\itemsep}{0pt}%
    \setlength{\parskip}{0pt}}%
{\end{itemize}}

\newenvironment{enumerate*}%
{\begin{enumerate}%
  \setlength{\itemsep}{0pt}%
    \setlength{\parskip}{0pt}}%
{\end{enumerate}}

\newenvironment{description*}%
{\begin{description}%
  \setlength{\itemsep}{0pt}%
    \setlength{\parskip}{0pt}}%
{\end{description}}

\begin{document}
\thispagestyle{empty}

\clearpage

\clearpage
\begin{center} 
%\bfseries{%
%%
%% ENTER TITLE OF PROPOSAL BELOW THIS LINE
{\large PROJECT SUMMARY}\\
%%
%%
%}
\end{center}

\begin{flushleft}
\noindent
{\bf \large Overview}\\
\grackle{} is an open-source library for computing the chemistry and
radiative cooling in astrophysical simulations and models, with
features that make it applicable to a wide variety of astrophysical
topics, including star, galaxy, and black hole formation, and the
evolution of the intergalactic medium.
The functionality that \grackle{} provides has traditionally
been implemented in isolated efforts by individual research groups,
with variations that make direct comparison of results difficult.
This type of machinery also requires frequent maintenance to
incorporate updated chemistry rates and new models.
\grackle{} exposes a full-featured chemistry and cooling solver
through an API that is straightforward to implement in simulation
codes written in C, C++, Fortran, and Python.
As a community project, \grackle{} performs two important services: 1)
it enables comparison and collaboration between codes through its universal API,
and 2) it provides a means for dissemination of new methods and
research results to the community by actively encouraging
contributions to the code.  \grackle{}'s success to date is evidenced by its
adoption in 14 different simulation codes.

We propose to undertake a series of development projects that are
critical to solidifying long-term community engagement in \grackle{}.
We will perform a necessary overhaul of the core routines to increase
modularity, to enable them to work with different solvers more easily,
and to be more developer-friendly.  We will upgrade \grackle{}'s
parallelism to support heterogeneous computing architectures including
GPU and many-core systems.  Using the enhanced framework created by
these improvements, we will then add new chemical networks that will
broaden its applicability to include new areas of research and expand
its user base.

\noindent
{\bf \large Intellectual Merit}\\
\grackle{} follows an established model for community development and
code releases that includes external review of source code changes
through pull requests, automated testing, and extensive user and
developer documentation.  The library exposes functionality equally to
simulation codes written in C, C++, Fortran, and Python with an API
that has been designed with careful consideration for maintaining
backward compatibility.  The Python interface is enhanced with
additional functionality for use in semi-analytic modeling.
\grackle{}'s functionality has been applied to a broad range of
astrophysical problems and this SSE will serve to further expand its
reach both through targeted feature additions and by enabling
community contribution.  The upgraded parallelism will enable
effective utilization of next-generation computing facilities and will
provide a model for developers as they add new features to the code.

\noindent
{\bf \large Broader Impacts}\\
\grackle{} provides equal access to functionality that is
necessary for many areas of research.  The Python interface's ease of
use lowers the barrier to entry to using tools that have
typically only been available in the context of large, complex
simulations.  By encouraging community contribution, \grackle{} gives
researchers a direct channel to its extensive user base, thus
providing an avenue for dissemination of work and for forming
collaborations.  By providing critical functionality equally to
simulation codes that use fundamentally different techniques (i.e.,
Eulerian, Lagrangian, and hybrid approaches), \grackle{}
also plays a key role in cross-platform collaborations and the
comparison of results.  This has already been demonstrated by the
AGORA simulation comparison project, which uses \grackle{} in nine
different simulation codes.  Finally, the project serves as an example
of the use of best practices for software development and computing
and provides a forum for less experienced developers to receive
guidance and mentorship through review of pull requests and
interaction with peers on the mailing list.

\end{flushleft}
\end{document}
