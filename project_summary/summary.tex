\documentclass[11pt]{article}
\usepackage{color}
\usepackage{setspace}
\usepackage{caption}
\usepackage{subcaption}
\usepackage{fancyvrb}
\usepackage{epsfig}
\usepackage{fullpage}
\usepackage[small,compact]{titlesec}
\usepackage{hyperref}
%\usepackage{times} 
\usepackage{wrapfig}
\usepackage{enumitem}
\setlist{nolistsep}
\setlength{\topmargin}{+0.0in}   %%%%%%%%% this is hacked (from +.0.1in) so that it looks right when converted.
\setlength{\oddsidemargin}{-0.0in}
\setlength{\evensidemargin}{-0.0in}
\setlength{\textheight}{9.0in}
\setlength{\textwidth}{6.5in}
\bibliographystyle{apj}
\newcommand{\apj}{ApJ}
\newcommand{\apss}{Astrophysics and Space Science}
\newcommand{\aj}{AJ}
\newcommand{\apjl}{ApJL}
\newcommand{\mnras}{MNRAS}
\newcommand{\apjs}{ApJS}
\newcommand{\pasp}{PASP}
\newcommand{\pasa}{PASA}
\newcommand{\araa}{ARA\&A}
\newcommand{\aap}{A\&A}
\newcommand{\aaps}{A\&AS}
\newcommand{\pasj}{PASJ}
\newcommand{\prd}{Phys. Rev. D}
\newcommand{\nat}{Nature}
\newcommand{\physrep}{Physics Reports}
\newcommand{\etal}{et al}
\newcommand{\grackle}{\texttt{Grackle}}
\newcommand{\dengo}{\texttt{Dengo}}
\newcommand{\yt}{\texttt{yt}}
\def\subsun{\mbox{$_{\odot}$}}
\def\lesssim{\mathrel{\hbox{\rlap{\hbox{%
 \lower4pt\hbox{$\sim$}}}\hbox{$<$}}}}
\def\gtrsim{\mathrel{\hbox{\rlap{\hbox{%
 \lower4pt\hbox{$\sim$}}}\hbox{$>$}}}}
\RequirePackage{natbib}
%\usepackage{setspace}
\setlength{\bibsep}{0.0pt}

\newenvironment{itemize*}%
{\begin{itemize}%
  \setlength{\itemsep}{0pt}%
    \setlength{\parskip}{0pt}}%
{\end{itemize}}

\newenvironment{enumerate*}%
{\begin{enumerate}%
  \setlength{\itemsep}{0pt}%
    \setlength{\parskip}{0pt}}%
{\end{enumerate}}

\newenvironment{description*}%
{\begin{description}%
  \setlength{\itemsep}{0pt}%
    \setlength{\parskip}{0pt}}%
{\end{description}}

\begin{document}
\thispagestyle{empty}

\clearpage

\clearpage
\begin{center} 
%\bfseries{%
%%
%% ENTER TITLE OF PROPOSAL BELOW THIS LINE
{\large PROJECT SUMMARY}\\
%%
%%
%}
\end{center}

\begin{flushleft}
\noindent
{\bf \large Overview}\\
\grackle{} is an open-source library for computing the chemistry and
radiative cooling in astrophysical simulations and models, with
features that make it applicable to a wide variety of astrophysical
topics, including star, galaxy, and black hole formation, and the
evolution of the intergalactic medium.
The functionality that \grackle{} provides has traditionally
been implemented in isolated efforts by individual research groups,
with variations that make direct comparison of results difficult.
This type of machinery also requires frequent maintenance to
incorporate updated chemistry rates and new models.
\grackle{} exposes a full-featured chemistry and cooling solver
through an API that is straightforward to implement in simulation
codes written in C, C++, Fortran, and Python.
As a community project, \grackle{} performs two important services: 1)
it enables comparison and collaboration between codes through its universal API,
and 2) it provides a means for dissemination of new methods and
research results to the community by actively encouraging
contributions to the code.
%% \grackle{}'s success to date is evidenced by its
%% adoption in 14 different simulation codes.

This proposal seeks to solidify long-term community engagement in
\grackle{} by overhauling the internal machinery to ease external
contribution, adding support for hardware acceleration, and
implementing additional functionality requested by the users. To
overcome the increasing difficulty and error rate of implementing more
complex solver, this project will integrate \grackle{} with a new package,
\dengo{}, that can automatically generate implicit solvers based on a
symbollically described network. \dengo{} will be upgraded to support
generation of explicit solvers, like \grackle{}, with hardware
acceleration support and ultimately generate an adaptive solver
capable of situation-aware method switching.

\noindent
{\bf \large Intellectual Merit}\\
\grackle{} follows an established model for community development and
code releases that includes external review of source code changes
through pull requests, automated testing, and extensive user and
developer documentation.  The library exposes functionality equally to
simulation codes written in C, C++, Fortran, and Python with an API
that has been designed with careful consideration for maintaining
backward compatibility.  Integration with the \dengo{} solver
generation code will allow for the construction of significantly more
complex chemical networks, opening up new applications in star and
planet formation, where observational data is outpacing simulations.
The functionality that will result from this award will be designed to
take advantage of modern hardware acceleration strategies so as to
make effective use of next-generation computing facilities.

%% The Python interface is enhanced with
%% additional functionality for use in semi-analytic modeling.
%% \grackle{}'s functionality has been applied to a broad range of
%% astrophysical problems and the proposed work will serve to further expand its
%% reach both through targeted feature additions and by enabling
%% community contribution.  The upgraded parallelism will enable
%% effective utilization of next-generation computing facilities and will
%% provide a model for developers as they add new features to the code.

\noindent
{\bf \large Broader Impacts}\\
\grackle{} provides equal access to functionality that is
necessary for many areas of research.  The Python interface's ease of
use lowers the barrier to entry to using tools that have
typically only been available in the context of large, complex
simulations.  By encouraging community contribution, \grackle{} gives
researchers a direct channel to its extensive user base, thus
providing an avenue for dissemination of work and for forming
collaborations.  By providing critical functionality equally to
simulation codes that use fundamentally different techniques (i.e.,
Eulerian, Lagrangian, and hybrid approaches), \grackle{}
also plays a key role in cross-platform collaborations and the
comparison of results.  This has already been demonstrated by the
AGORA simulation comparison project, which uses \grackle{} in nine
different simulation codes.  Integration of the \dengo{} solver
generation package will allow for the implementation of much more
complex networks, expanding \grackle{'s} user base and providing a
means of accessing published networks with no publicly available
code.

%% As a well-tested, thoroughly documented library with a well-designed
%% structure and clear API, \grackle{} also provides an example of the
%% use best practices for software development and computing.  This is
%% furthered by the guidance and mentorship that less experienced
%% developers receive through reviews of Pull Requests and
%% advice on the mailing list.  These skills can be applied far
%% beyond computational astrophysics, in other sciences and industry.  Integration
%% with the technologies underpinning \dengo{}, such as jinja2, sympy and ODE
%% solver systems, will help to integrate researchers into a modern software stack
%% that is being used across disciplines, from software engineering and web
%% development to data science and theoretical mathematics.

\end{flushleft}
\end{document}
